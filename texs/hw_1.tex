\documentclass[a4paper,12pt]{article}
\usepackage{amssymb}
\usepackage[mathcal]{euscript}
\usepackage{graphicx}
\usepackage{amsmath}
\usepackage{hyperref}
\usepackage[utf8]{inputenc}
\usepackage[russian]{babel}
\begin{document}
\title{Домашнее задание №1}
\author{Дудырев Егор}
\date{\today}
\maketitle

\section{Задание}
\textbf{Задание D}\\
Выполнить лексико-статистический анализ двух текстов на русском языке, среднего размера:
\begin{enumerate}
\item текст художественной прозы, например, текста главы 2 из книги Л Кэрролла «Алиса в Стране Чудес» или главы из романа Л.Толстого или рассказа А.Чехова.
\item текст научно-технической или деловой прозы.
\end{enumerate}

Для этого следует составить программу, которая с помощью выбранного морфопроцессора, например: Диалинг-АОТ , mystem, pymorphy или CrossMorphy (ссылки см. в варианте А), осуществляет морфологический анализ словоформ текста; вычисляет 7-15 статистических характеристик разного типа:
\begin{enumerate}
\item общестатистические: общее число словоупотреблений, число различных словоформ, средняя длина предложения (если процессор разбивает текст на предложения) и т.п.;
\item морфологические: абсолютная и относительная частота омонимичных словоформ, процент разных частей речи, наиболее частотные падежи у существительных и прилагательных, относительную частоту падежей, наиболее частотные морфологические формы глаголов (время/лицо/число) и т.п.;
\item лексические: количество уникальных лемм, число уникальных лемм разных частей речи (существительных, глаголов и др.), число незнакомых слов, самые частотные слова и их относительная частота, самые частотные слова основных частей речи (существительные, прилагательные, наречия, глаголы), коэффициент лексического богатства текста (=отношение числа различных лемм к общему числу словоупотреблений) и т.п. выводит подсчитанные характеристики в удобной, обозримой форме.
\end{enumerate}

Отчет: составленная программа, подсчитанная статистика (в удобной, обозримой форме, в зависимости от стилей/жанров текстов), пояснения по способу ее подсчета и выводы, программа с комментариями.

\section{Используемые данные}
Сравнение двух (художественного и научно-делового) текстов:
\begin{itemize}
\item Худ. текст: Глава 2 книги "Алиса в Стране Чудес" Л. Кэрролла
\item Деловой текст: Глава 1 книги "Пособие по написанию разного рода деловых текстов" Г.П. Несговорова
\end{itemize}

Сравнение нескольких художественных и научно-деловых текстов:
\begin{itemize}
\item Художественные тексты
\begin{enumerate}
\item Книги "Алиса в Стране Чудес" Л. Кэрролла
\item Рассказ "Крыжовник" А.П. Чехова
\item Рассказ "Радость" А.П. Чехова
\item Глава 1 книги "Трудно быть богом" братьев Стругацких
\item Глава 1 книги "Детство" Л.Н. Толстого
\end{enumerate}
\item Научно-деловые тексты
\begin{enumerate}
\item Главы 1,2 книги "Пособие по написанию разного рода деловых текстов" Г.П. Несговорова
\item Статья "Компьютерная лингвистика: Теория и практика" М.Р. Смагина
\item Статья "Алгоритм помог расслышать гравитационные волны от землетрясений без суперЭВМ" сайта N+1
\item Статья "Антиматерия не отличилась от материи взаимодействием с квантовыми флуктуациями" сайта N+1
\item Статья "Универсальность математики" Ф.Х. Айматовой
\end{enumerate}
\end{itemize}

\section{Используемое ПО}
Язык программирования: Python 3.7 (с дистрибутивом Anaconda).\\
В качестве лексического анализатора использовалась библиотека pymorphy2 0.8\\
Используемые библиотеки:
\begin{itemize}
\item numpy 1.17.2
\item pandas 0.25.1
\item matplotlib 3.1.1
\item seaborn 0.9.0
\item re 2.2.1
\item sklearn 0.21.3
\end{itemize}

\section{Сравнение научного и художественного текстов}
Сильнее всего среди текстов различаются следующие характеристики слов:
\begin{itemize}
\item Вид глагола: В научном тексте много глаголов несовершенного вида, в художественном - совершенного
\item Падежи: В научном тексте половина слов имеет родительный падеж, в художественном - именительный
\item Лицо глагола: В научном тексте абсолютное большинсто глаголов - в 3ем лице. В художественном тексте анализатор pymorphy2 не может определить лица половины глаголов - вероятно они своеобразно, художественно используются.
\item Части речи: В научном тексте много существительных и прилагательных, в художественном - глаголов и местоимений.
\item Время глагола: В научном тексте абсолютное большинство глаголов указаны в настоящем времени. В художественном - очень много глаголов в прошедшем времени.
\item Среднее количество слов в предложении научного текста примерно в 3 раза больше, чем в художественном.
\item Наиболее частые леммы текстов отличаются, но являются специфичными для конкретного текста. Например в "Пособии по написанию ... текстов" очень часто встречается лемма "язык", а в "Алисе в Стране Чудес" - имя "Алиса".
\end{itemize}

\section{Сравнение нескольких текстов}
Сравнить между собой два текста интересно, но ещё интереснее было бы сравнить абстрактные художественный и научно-деловой текст.\\
Для этого загрузим несколько различных текстов различных авторов (русские и английский художественные тексты, научные тексты по лингвистике, математике, физике), разобьём их на предложения и сгенерируем случайным образом из всех полученных предложений несколько абстрактных художественных и научно-деловых текстов. Затем обучим на них модель машинного обучения (в частности Деревья решений) и найдём главные отличия двух типов текстов.\\
В результате получились следующие правила, способные со стопроцентной точностью отделить художественный текст от научного:
\begin{enumerate}
\item Доля глаголов несов. вида > 0.67 => науч.-деловой
\item Доля глаголов сов. вида > 0.33 => художественный
\item Доля слов в род. падеже > 0.38 => науч.-деловой
\item Доля слов в 3ем лице > 0.42 => науч.-деловой
\item Доля слов с нераспознанным pymorphy2 лицом > 0.50 => художественный
\item Доля полных прилагательных > 0.15 => науч.-деловой
\item Доля существительных > 0.34 => науч.-деловой
\item Доля мест.-существительных > 0.04 => художественный
\item Доля глаголов > 0.11 => художественный
\item Доля слов в прошедшем времени > 0.47 => художественный
\item Доля слов в настоящем времени > 0.43 => науч.-деловой
\item Кол-во уникальных лемм полн. прилагательных > 85.00 => науч.-деловой
\end{enumerate}

\section{Ссылки на код}
\begin{itemize}
\item \href{http://localhost:8888/notebooks/notebooks/hw1/hw1_24.02.2020-Final.ipynb}{Исполняемый код}
\item \href{https://github.com/EgorDudyrev/hse_comp_ling/tree/master/imgs}{Графики сравнения двух текстов}
\end{itemize}

\end{document}